

\documentclass[12pt]{article}
\usepackage[utf8]{inputenc}
\usepackage{marvosym}
\usepackage{color}
\usepackage{amssymb,amsmath,amstext,amsfonts,amsthm}
\usepackage[gen]{eurosym}
\usepackage{fancyhdr,fancyvrb}
\usepackage{latexsym}
\usepackage{graphicx}
\usepackage{placeins,lscape}
\usepackage[normalem]{ulem}
\usepackage{cancel}
\usepackage{chngcntr}
\usepackage{footnote}
\usepackage{caption}
\usepackage{array}
\usepackage{rotating}
\usepackage{dcolumn }
\captionsetup[figure]{labelfont=bf}
\captionsetup[table]{labelfont=bf}
\usepackage{subfigure}
\usepackage{colortbl}
\newcommand{\gray}{\rowcolor[gray]{.90}}
\setlength{\parindent}{0cm}
\renewcommand{\rmdefault}{ptm}
\renewcommand{\thesubsection}{\thesection.\alph{subsection}}
\usepackage{alltt}
\usepackage{lscape}
\usepackage{dcolumn}
\usepackage{booktabs,caption,fixltx2e}
\usepackage[flushleft]{threeparttable}
\usepackage{float}
\usepackage{graphicx}
\captionsetup[figure]{skip=0pt}
\usepackage[paper=portrait,pagesize]{typearea}
\usepackage{setspace}
\usepackage{lmodern}
\DeclareMathOperator{\EX}{\mathbb{E}}% expected value
\setcounter{MaxMatrixCols}{10}
\setcounter{tocdepth}{5}
\setcounter{secnumdepth}{5}
\oddsidemargin 0cm \evensidemargin 0mm \topmargin -20mm \textwidth 16.5cm \textheight 24cm
\setlength{\parskip}{5pt}
\renewcommand{\topfraction}{.85}
\renewcommand{\bottomfraction}{.7}
\renewcommand{\textfraction}{.15}
\renewcommand{\floatpagefraction}{.66}
\renewcommand{\dbltopfraction}{.66}
\renewcommand{\dblfloatpagefraction}{.66}
\setcounter{topnumber}{9}
\setcounter{bottomnumber}{9}
\setcounter{totalnumber}{20}
\setcounter{dbltopnumber}{9}
\newtheorem{taable}{Table}
\renewcommand{\baselinestretch}{1.5}
\linespread{1.5}
\usepackage{listings}
\inputencoding{latin1}
\inputencoding{utf8}
\newcommand{\vect}[1]{\boldsymbol{#1}}
%\usepackage{tgpagella} % text only
\usepackage{mathpazo}  % math & text


%%%%%%%%%%%%%%%%%%%% CITE %%%%%%%%%%%%%%%%%%%%%%%%%%%%
\usepackage{hyperref}
\hypersetup{
	colorlinks=true,
	linkcolor=red,
	filecolor=magenta,      
	urlcolor=cyan,
}
\usepackage[round, sort, authoryear]{natbib}
\bibliographystyle{apalike}
\hypersetup{citecolor=red}
\usepackage[numbib]{tocbibind}

%%%%%%%%%%%%%%%%%%%%%%%%%%%%%%%%%%%%%%%%%%%%%%%%%%%%%


\title{Dealing with the Dutch Cap Bonus \thanks{Acknowledgments: All remaining errors are ours.}}

\author{Ata Bertay \thanks{%
		Sabanci Business School. Sabanci University.}   \ \   José Gabo Carreño\thanks{%
		Department of economics. Tilburg University} \ \ \ Harry Huizinga\footnotemark[3] \\  Burak Uras\footnotemark[3] \ \ Nathanael Vellekoop\thanks{Department of economics. University of Toronto.} }


\date{\today}



\begin{document}

\maketitle

%\centerline{[\textbf{\textit{PRELIMINARY VERSION}}]}
\begin{abstract}

We use a ...

\textbf{Keywords}: ...

\end{abstract}
\vspace{1cm}


\newpage

%===================================================================================
\section{Introduction}
%===================================================================================
....




\newpage
%=====================================================================================
\section{Literature Review} \label{literature_review}
%=====================================================================================





\begin{itemize}
	
	\item \citet{berger2020stimulating}: They document the effect of the FTHC on home sales. Our difference in differences design compares ZIP codes at the same point in time whose
	exposure to the program differs. We define program exposure based on the number of potential first-time home buyers in a ZIP code. ZIP codes with few potential first-time homebuyers serve as a “control group” because the policy does not induce many people to buy in these places. We measure exposure as
	the year-2000 share of people in a ZIP code who are first-time homebuyers. 
	
	\begin{figure}
		\centering
		\caption{Figure 5 from \citet{berger2020stimulating}.}
		\includegraphics[width=0.7\linewidth]{figure5_empirical_approach}
		
		\label{fig:figure5empiricalapproach}
	\end{figure}
	
	
	
	\item \citet{colonnello2018effectiveness}:  A very good starting point to start working on this topic. They classified bankers with a binding bonus cap as treated.  They then run a DiD where the outcome is risk-taking.  They find mixed evidence. He also seems that literature does not have evidence for the full economy to study the impact of bonus cap in the finance industry (except by \citet{abudy2020executive}).   
	\item \citet{efing2018bank}: They argue that risk sharing motivates the bank-wide structure of bonus pay. This is paper is interesting because they have employee-employer data. Yet do not have advantage of any quasi-experiment but the financial crisis.  They document very interesting fact that we can also look at. This paper is motivated in the theoretical work of \citet{thanassoulis2012case}.
	\item \citet{abudy2020executive}: They exploit a the same quasi-experiment than us.  They find that compensation
	contracts can be set in a way that does not maximize firm value. Yet, they have information at firm level. We have information employeer-employee. 
	
\end{itemize}








%=====================================================================================
\section{Data} \label{data}
%=====================================================================================


[DESCRIPTION OF THE DUTCH DATA]


\subsection{Timing of the bonus cap}
	
	
	\begin{itemize}
		\item \textbf{On 5 March 2013} the EU finance ministers decided that European banker's bonuses should be capped at a maximum of 100\% their base salary, rising to 200\% if shareholders explicitly agree. 
		\item \textbf{On 26 November 2013}, the Dutch government published a draft legislative proposal in which it introduced a 20\% bonus
		cap for financial undertakings. 
		\item \textbf{On 7 February 2015} was introduced the Dutch Act on the Remuneration Policies Financial Undertakings (Wet beloningsbeleid financiële ondernemingen, the ``Act'').
		\item The central point of the Act is the 20\% bonus cap: a financial undertaking cannot pay any person ``working under its responsibility'' variable remuneration that exceeds 20\% of the fixed remuneration on an annual
		basis.\footnote{Remuneration is either fixed or variable. The Act has a broad definition of variable remuneration, ``all remuneration that is not fixed remuneration'', whereas the definition of fixed remuneration is ``the part of the total remuneration that consists of unconditional financial or non-financial payments''.}
		\item Employees may be awarded a bonus exceeding 20\% of the fixed pay in 2015 if such award stems from an obligation existing prior to 1 January 2015. The Minister of Finance has made it clear that this exception only applies to 2014 performance bonuses that are awarded in 2015.\textbf{ As from 1 January 2016, any bonus award is subject to the rules of the Act.}
		\item The Act applies to financial undertakings (financiële ondernemingen) with their official seat in the Netherlands and their subsidiaries (including subsidiaries abroad).
		\item The definition of ``financial undertaking'' is very broad and includes, amongst others, banks, insurers, investment firms, fund managers, payment services providers, custodians and premium pension institutions (PPIs). \textbf{However, the bonus cap does not apply to:}
		\begin{itemize}
			\item Pension funds.
			\item Asset managers.
			\item  Branches of banks and investment
			firms located in the Netherlands with a
			‘mother firm’ in another member state
			of the European Union and which are
			subject to the Capital Requirements
			Directives IV (CRD IV).
			\item Investment institutions and institutions
			for collective investment in securities.
			These institutions should invest on their
			own account with their own resources
			and capital, and have no external
			customers.
		\end{itemize}
		
	\end{itemize}
	
	
\subsection{Evidence looking at the data}
	
	\begin{itemize}
		\item Our data allows us to decompose the worker's pay into base salary, special remuneration,\footnote{This concerns, for example, holiday allowance, end-of-year benefits, performance benefits, bonuses and profit distributions. The special rewards do not include: contributions to savings schemes, termination benefits, health insurance allowances and wages for overtime.} extra salaries,\footnote{Regarding the extra salary, it is not clear if this item should be considered fixed or variable. We can show that banks used it to compensate bankers in 2014.} and overtime worked hours. 
		\item We define the variable pay ratio as
		\begin{equation}
			VPR = \frac{\text{Special remuneration}}{\text{basic wage}}.
		\end{equation}
		\item We work with just two three-industries: Banks, Insurance services, and Pension Services.  
		\item Regarding the VPR dynamics, we have the following. 
		\begin{enumerate}
			\item Banks: While the VPR jumped in 2013, it decreased in 2014 to a level lower than 2012. After 2014, the VPR has remained in approximately the same level than 2014. Therefore, the dynamic are related to the announcement, but not to the start of the regulation.\footnote{The lower VPR is explained by a lower special remuneration. The basic wage has grown constantly over time. } \textbf{Who were more affected?}  Worker on the top decile according to the gross wage. It seems that they were also more prone to leave the finance industry. 
			\item Banks: The VPR in 2012 cannot help to explain who were treated in 2014. Two reasons: (1) the VPR is not related to the gross wage. Everyone can get a high VPR in the finance industry; (2) The VPR in in fact variable. Getting a high VPR in 2012 does not predict a higher VPR in 2014. 
			%\item Small exercise. We consider workers staying in the finance industry over the period 2012-2014. We classify workers in deciles by using the gross wage in 2012. We calculate the VPR change between 2012 and 2014. We do see that workers at the top deciles were more affected by the dutch cap. They suffered a 2.5\% media reduction of the VPR between 2012 and 2014. We also observe a reduction in the gross wage of workers at the top deciles. 
			%\item It is important to note that there is not a clear relationship between the VPR and the gross wage. So, it is not possible to anticipate
			%\item Continuing the exercise, we check how many workers were still employed in 2015. Remember that we are considering workers stayed in the finance industry over the period 2012-2014. We do see that 20\% of workers at the top decile left the finance industry. Contrarily, we see that just 10\% of worker left at lower deciles. 
			
		\end{enumerate}
		
	\end{itemize}
	



\subsection{Summary Statistics}


[TO COMPLETE]




%=====================================================================================
\section{Estimation Methods} \label{estimation_methods}
%=====================================================================================

To estimate 

\newpage
\section{Concluding remarks} \label{conclusions}

This paper has analyzed the 


\newpage





\setcitestyle{numbers}
%\bibliographystyle{apalike}
\bibliography{BCHUV_cap_biblio_2022}





\newpage
























\appendix
\renewcommand{\thefigure}{A\arabic{figure}}
\renewcommand{\thetable}{A\arabic{table}}
\renewcommand{\theequation}{A\arabic{equation}}

\setcounter{figure}{0}
\setcounter{table}{0}

%EXAMPLE OF HOW TO USE SUBFIGURE (DO NOT DELETE)
%==============================================================================
%\begin{figure}[H] 
%\centering 
%\caption{Wage rigidities and welfare under CM2016's loss function: currency union versus inflation targeting.} \label{fig:gm2016_welfare_losses} 
%\subfigure[Demand shocks]{% 
%	\includegraphics[width=.45\textwidth]{wealth_loss_infl_vs_peg_z_shock_CM2016} } 
%\quad
%\subfigure[Technology shocks]{\includegraphics[width=.45\textwidth]{wealth_loss_infl_vs_peg_a_shock_CM20%16} } 
%
%\end{figure}
%==============================================================================


\newpage
\KOMAoptions{paper=portrait,pagesize}
\recalctypearea

\newpage
\begin{center}
    \textbf{APPENDIX}
\end{center}

\section{XXXX}


\end{document}
